\documentclass[12pt]{article}
\usepackage[top=1in,bottom=1in,left=0.75in,right=0.75in,centering]{geometry}
\usepackage{fancyhdr}
\usepackage{epsfig}
\usepackage[pdfborder={0 0 0}]{hyperref}
\usepackage{palatino}
\usepackage{wrapfig}
\usepackage{lastpage}
\usepackage{color}
\usepackage{ifthen}
\usepackage[table]{xcolor}
\usepackage{graphicx,type1cm,eso-pic,color}
\usepackage{hyperref}
\usepackage{amsmath}
\usepackage{wasysym}
\usepackage{latexsym}
\usepackage{amssymb}

\def\course{CS 4102: Algorithms}
\def\homework{Module 2 - Graphs: Basic Written HW}
\def\semester{Spring 2021}

\newboolean{solution}
\setboolean{solution}{false}

% add watermark if it's a solution exam
% see http://jeanmartina.blogspot.com/2008/07/latex-goodie-how-to-watermark-things-in.html
\makeatletter
\AddToShipoutPicture{%
\setlength{\@tempdimb}{.5\paperwidth}%
\setlength{\@tempdimc}{.5\paperheight}%
\setlength{\unitlength}{1pt}%
\put(\strip@pt\@tempdimb,\strip@pt\@tempdimc){%
\ifthenelse{\boolean{solution}}{
\makebox(0,0){\rotatebox{45}{\textcolor[gray]{0.95}%
{\fontsize{5cm}{3cm}\selectfont{\textsf{Solution}}}}}%
}{}
}}
\makeatother

\pagestyle{fancy}

\fancyhf{}
\lhead{\course}
\chead{Page \thepage\ of \pageref{LastPage}}
\rhead{\semester}
%\cfoot{\Large (the bubble footer is automatically inserted into this space)}

\setlength{\headheight}{14.5pt}

\newenvironment{itemlist}{
\begin{itemize}
\setlength{\itemsep}{0pt}
\setlength{\parskip}{0pt}}
{\end{itemize}}

\newenvironment{numlist}{
\begin{enumerate}
\setlength{\itemsep}{0pt}
\setlength{\parskip}{0pt}}
{\end{enumerate}}

\newcounter{pagenum}
\setcounter{pagenum}{1}
\newcommand{\pageheader}[1]{
\clearpage\vspace*{-0.4in}\noindent{\large\bf{Page \arabic{pagenum}: {#1}}}
\addtocounter{pagenum}{1}
\cfoot{}
}

\newcounter{quesnum}
\setcounter{quesnum}{1}
\newcommand{\question}[2][??]{
\begin{list}{\labelitemi}{\leftmargin=2em}
\item [\arabic{quesnum}.] {#2}
\end{list}
\addtocounter{quesnum}{1}
}


\definecolor{red}{rgb}{1.0,0.0,0.0}
\newcommand{\answer}[2][??]{ 
\ifthenelse{\boolean{solution}}{
\color{red} #2 \color{black}}
{\vspace*{#1}}
}

\definecolor{blue}{rgb}{0.0,0.0,1.0}

\begin{document}

\section*{\homework}


%----------------------------------------------------------------------

\question[3]{
Let $G$ be an undirected graph with $n$ nodes (let's assume $n$ is even). Prove or provide a counterexample for the following claim: If every node of $G$ has a degree of at least $\frac{n}{2}$, then $G$ must be connected.
}



\question[3]{
Most graph algorithms that take an adjacency-matrix representation as input require time $\Omega(V^2)$, but there are some exceptions. Show how to determine whether a directed graph G contains a universal sink—a vertex with in-degree $|V| - 1$ and out-degree $0$ in time $O(V)$, given an adjacency matrix for G.
}



%----------------------------------------------------------------------

\question[1]{
The textbook describes two variables that can be associated with each node in a graph $G$ during the execution of \emph{Depth-First Search}: discovery time ($v.d$) and finish time ($v.f$). These are integer values that are unique. Every time a node is discovered (i.e., DFS sees the node for the first time) that node's $v.d$ is set to the next available integer. When DFS is finished exploring ALL of this node's children, $v.f$ is set to the next available integer.

For this question, consider a single edge in a graph $G$ after DFS finishes executing. You might need to reference the textbook or slides for definitions of tree edge, forward edge, back edge, and cross edge. Argue that each edge $e=(u,v)$ is:

\begin{enumerate}
\item A tree edge or forward edge if and only if $u.d < v.d < v.f < u.f$
\item A back edge if and only if $v.d \leq u.d < u.f \leq v.f$
\item A cross edge if and only if $v.d < v.f < u.d < u.f$
\end{enumerate}

You can describe your answers intuitively, but your answers must be clearly articulated.
}



%----------------------------------------------------------------------

\question[1]{
\emph{Kruskal's algorithm} begins by adding the smallest edge in the graph to the solution (and never looking back). Let $e=(u,v)$ be a minimum-weight edge in a connected graph $G$. Show that $e=(u,v)$ belongs to some minimum spanning tree of G. \emph{HINT: Use a proof by contradiction. Note that $e$ will eventually connect two smaller spanning trees together. If $e$ is NOT in the solution, than something else IS. Show that this leads to some kind of contradiction.}
}





\end{document}
