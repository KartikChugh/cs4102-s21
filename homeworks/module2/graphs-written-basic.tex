\documentclass[12pt]{article}
\usepackage[top=1in,bottom=1in,left=0.75in,right=0.75in,centering]{geometry}
\usepackage{fancyhdr}
\usepackage{epsfig}
\usepackage[pdfborder={0 0 0}]{hyperref}
\usepackage{palatino}
\usepackage{wrapfig}
\usepackage{lastpage}
\usepackage{color}
\usepackage{ifthen}
\usepackage[table]{xcolor}
\usepackage{graphicx,type1cm,eso-pic,color}
\usepackage{hyperref}
\usepackage{amsmath}
\usepackage{wasysym}
\usepackage{latexsym}
\usepackage{amssymb}

\def\course{CS 4102: Algorithms}
\def\homework{Module 2 - Graphs: Basic Written HW}
\def\semester{Spring 2021}

\newboolean{solution}
\setboolean{solution}{false}

% add watermark if it's a solution exam
% see http://jeanmartina.blogspot.com/2008/07/latex-goodie-how-to-watermark-things-in.html
\makeatletter
\AddToShipoutPicture{%
\setlength{\@tempdimb}{.5\paperwidth}%
\setlength{\@tempdimc}{.5\paperheight}%
\setlength{\unitlength}{1pt}%
\put(\strip@pt\@tempdimb,\strip@pt\@tempdimc){%
\ifthenelse{\boolean{solution}}{
\makebox(0,0){\rotatebox{45}{\textcolor[gray]{0.95}%
{\fontsize{5cm}{3cm}\selectfont{\textsf{Solution}}}}}%
}{}
}}
\makeatother

\pagestyle{fancy}

\fancyhf{}
\lhead{\course}
\chead{Page \thepage\ of \pageref{LastPage}}
\rhead{\semester}
%\cfoot{\Large (the bubble footer is automatically inserted into this space)}

\setlength{\headheight}{14.5pt}

\newenvironment{itemlist}{
\begin{itemize}
\setlength{\itemsep}{0pt}
\setlength{\parskip}{0pt}}
{\end{itemize}}

\newenvironment{numlist}{
\begin{enumerate}
\setlength{\itemsep}{0pt}
\setlength{\parskip}{0pt}}
{\end{enumerate}}

\newcounter{pagenum}
\setcounter{pagenum}{1}
\newcommand{\pageheader}[1]{
\clearpage\vspace*{-0.4in}\noindent{\large\bf{Page \arabic{pagenum}: {#1}}}
\addtocounter{pagenum}{1}
\cfoot{}
}

\newcounter{quesnum}
\setcounter{quesnum}{1}
\newcommand{\question}[2][??]{
\begin{list}{\labelitemi}{\leftmargin=2em}
\item [\arabic{quesnum}.] {#2}
\end{list}
\addtocounter{quesnum}{1}
}


\definecolor{red}{rgb}{1.0,0.0,0.0}
\newcommand{\answer}[2][??]{ 
\ifthenelse{\boolean{solution}}{
\color{red} #2 \color{black}}
{\vspace*{#1}}
}

\definecolor{blue}{rgb}{0.0,0.0,1.0}

\begin{document}

\section*{\homework}


%----------------------------------------------------------------------
\question[3]{
Describe an algorithm that, given a directed graph G represented as an \emph{adjacency matrix}, returns whether or not the graph contains vertex with in-degree $|V| - 1$ and out-degree $0$. In other words, does the graph have a node such that every other node points to it, but it does not point to any other node. Your algorithm must be $O(V)$. Note that there are $\Theta(V^2)$ cells in your adjacency matrix so you'll need to be clever here.
}

\answer[0.5in]{...}

\question[1]{
Let's say a graph $G$'s {\em circumference} is the number of edges in the shortest cycle in $G$. Describe an efficient algorithm to find the circumference of a graph in $\Theta(V \times E)$.  (Note: you must make use of algorithms studied in this module, and not  re-invent the wheel.)
}

\answer[0.5in]{...}

\question[3]{
Let $G$ be an undirected graph with $n$ nodes (let's assume $n$ is even). Prove or provide a counterexample for the following claim: If every node of $G$ has a degree of at least $\frac{n}{2}$, then $G$ must be connected.
}

\answer[0.5in]{...}

\question[1]{
 For a given undirected graph $G$, prove that the depth of a DFS tree cannot be smaller than the depth of the BFS tree.  (Clearly state your proof strategy or technique.)
}

% ------------------------------------------



% ------------------------------------------



% ------------------------------------------



\iffalse

\question[1]{
In the CLRS code for {\em DFS-Visit} at line 8,  we have completed recursively visiting all then un-visited vertices adjacent to vertex $u$, so we change $u$'s color to black.  For this problem, consider what happens if instead we change the color of $u$ to {\em white} at this line.

Explore what this causes to happen by tracing this altered algorithm on a small connected undirected graph.  After you have done that, submit answers to the following questions.

\begin{enumerate}
\renewcommand{\theenumi}{\Alph{enumi}}
\item In no more than a few sentences, clearly describe what the ``DFS'' tree from an initial node $s$ found by this algorithm represents.  (Note:  we put DFS in quotes here because with this change this algorithm is no longer DFS!)
\item Give a mathematical formula (not just an order class) for the number of leaves in this tree for a worst-case input graph.  Describe this worst-case input graph.  (Again, tracing this algorithm on small connected graphs may help you find the answer more quickly.)
\end{enumerate}

}

\question[1]{
The textbook describes two variables that can be associated with each node in a graph $G$ during the execution of \emph{Depth-First Search}: discovery time ($v.d$) and finish time ($v.f$). These are integer values that are unique. Every time a node is discovered (i.e., DFS sees the node for the first time) that node's $v.d$ is set to the next available integer. When DFS is finished exploring ALL of this node's children, $v.f$ is set to the next available integer.

For this question, consider a single edge in a graph $G$ after DFS finishes executing. You might need to reference the textbook or slides for definitions of tree edge, forward edge, back edge, and cross edge. Argue that each edge $e=(u,v)$ is:

\begin{enumerate}
\item A tree edge or forward edge if and only if $u.d < v.d < v.f < u.f$
\item A back edge if and only if $v.d \leq u.d < u.f \leq v.f$
\item A cross edge if and only if $v.d < v.f < u.d < u.f$
\end{enumerate}

You can describe your answers intuitively, but your answers must be clearly articulated.
}



\question[1]{
Given an undirected graph $G$, the {\em eccentricity} of a node $v$ is the largest of the shortest possible distances from $v$ to any other node in the graph. The minimum eccentricity in the graph is called the  {\em graph radius} for $G$. All the nodes in $G$ that have eccentricity equal to the graph radius form a set called the {\em graph center} of G.

Describe (using pseudo-code or a very clea rtext explanation) an efficient algorithm to find the graph center of a graph $G$ and describe its complexity.  (Note: you must make use of algorithms studied in this module, and not  re-invent the wheel.)
}
\fi


\end{document}
