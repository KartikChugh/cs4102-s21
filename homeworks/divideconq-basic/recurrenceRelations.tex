\documentclass[12pt]{article}
\usepackage[top=1in,bottom=1in,left=0.75in,right=0.75in,centering]{geometry}
\usepackage{fancyhdr}
\usepackage{epsfig}
\usepackage[pdfborder={0 0 0}]{hyperref}
\usepackage{palatino}
\usepackage{wrapfig}
\usepackage{lastpage}
\usepackage{color}
\usepackage{ifthen}
\usepackage[table]{xcolor}
\usepackage{graphicx,type1cm,eso-pic,color}
\usepackage{hyperref}
\usepackage{amsmath}
\usepackage{wasysym}

\def\course{CS 4102: Algorithms}
\def\homework{Divide and Conquer / Sorting Basic: Recurrence Relations}
\def\semester{Spring 2021}

\newboolean{solution}
\setboolean{solution}{false}

% add watermark if it's a solution exam
% see http://jeanmartina.blogspot.com/2008/07/latex-goodie-how-to-watermark-things-in.html
\makeatletter
\AddToShipoutPicture{%
\setlength{\@tempdimb}{.5\paperwidth}%
\setlength{\@tempdimc}{.5\paperheight}%
\setlength{\unitlength}{1pt}%
\put(\strip@pt\@tempdimb,\strip@pt\@tempdimc){%
\ifthenelse{\boolean{solution}}{
\makebox(0,0){\rotatebox{45}{\textcolor[gray]{0.95}%
{\fontsize{5cm}{3cm}\selectfont{\textsf{Solution}}}}}%
}{}
}}
\makeatother

\pagestyle{fancy}

\fancyhf{}
\lhead{\course}
\chead{Page \thepage\ of \pageref{LastPage}}
\rhead{\semester}
%\cfoot{\Large (the bubble footer is automatically inserted into this space)}

\setlength{\headheight}{14.5pt}

\newenvironment{itemlist}{
\begin{itemize}
\setlength{\itemsep}{0pt}
\setlength{\parskip}{0pt}}
{\end{itemize}}

\newenvironment{numlist}{
\begin{enumerate}
\setlength{\itemsep}{0pt}
\setlength{\parskip}{0pt}}
{\end{enumerate}}

\newcounter{pagenum}
\setcounter{pagenum}{1}
\newcommand{\pageheader}[1]{
\clearpage\vspace*{-0.4in}\noindent{\large\bf{Page \arabic{pagenum}: {#1}}}
\addtocounter{pagenum}{1}
\cfoot{}
}

\newcounter{quesnum}
\setcounter{quesnum}{1}
\newcommand{\question}[2][??]{
\begin{list}{\labelitemi}{\leftmargin=2em}
\item [\arabic{quesnum}.] {} {#2}
\end{list}
\addtocounter{quesnum}{1}
}


\definecolor{red}{rgb}{1.0,0.0,0.0}
\newcommand{\answer}[2][??]{
\ifthenelse{\boolean{solution}}{
\color{red} #2 \color{black}}
{\vspace*{#1}}
}

\definecolor{blue}{rgb}{0.0,0.0,1.0}

\begin{document}

\section*{\homework}


\question[3]{
You are a hacker, trying to gain information on a secret array of size $n$. This array contains $n-1$ ones and exactly $1$ two; you want to determine the index of the two in the array.\\
\\
Unfortunately, you don't have access to the array directly; instead, you have access to a function $f(l1, l2)$ that compares the sum of the elements of the secret array whose indices are in $l1$ to those in $l2$. This function returns $-1$ if the $l1$ sum is smaller, $0$ if they are equal, and $1$ if the sum corresponding to $l2$ is smaller.\\
\\
For example, if the array is $a=[1,1,1,2,1,1]$ and you call $f([1,3,5],[2,4,6])$ then the return value is $1$ because $a[1]+a[3]+a[5]=3<4=a[2]+a[4]+a[6]$. Design an algorithm to find the index of the $2$ in the array using the least number of calls to $f()$. Suppose you discover that $f()$ runs in $\Theta(max(|l1|,|l2|))$, what is the overall runtime of your algorithm? 
}

\vspace{12pt}

\question[3]{
In class, we looked at the \emph{Quicksort algorithm}. Consider the \textbf{worst-case scenario} for quick-sort in which the worst possible pivot is chosen (the smallest or largest value in the array). Answer the following questions:

\begin{itemize}
\item What is the probability of choosing one of the two worst pivots out of $n$ items in the list?
\item Extend your formula. What is the probability of choosing the one of the worst possible pivots \emph{for EVERY recursive call} until reaching the base case. In other words, what is the probability quicksort fully sorts the list while choosing the worst pivot choice every time it attempts to do so?
\item What is the limit of your formula above as the size of the list grows. Is the chance of getting Quicksort's worst-case improving, staying constant, or converging on some other value.
\item Present one sentence on what this means. What are the chances that we actually get Quicksort's worst-case behavior?
\end{itemize}
}

\vspace{12pt}



%----------------------------------------------------------------------
\noindent Directly solve, by unrolling the recurrence, the following relation to find its exact solution.

\question[2]{
$T(n) = T(n-1) + n$
}

\vspace{12pt}

%----------------------------------------------------------------------

\noindent Use induction to show bounds on the following recurrence relations.

\question[2]{
Show that $T(n)=2T(\sqrt{n})+log(n) \in O(log(n)*log(log(n)))$. \emph{Hint: Try creating a new variable m and substituting the equation for m to make it look like a common recurrence we've seen before. Then solve the easier recurrence and substitute n back in for m at the end.}
}

\answer[0 in]{
...
}

\question[2]{
Show that $T(n)=4T(\frac{n}{3})+n \in \Theta(n^{log_3(4)})$. You'll need to subtract off a lower-order term to make the induction work here. \emph{Note: we are using big-theta here, so you'll need to prove the upper AND lower bound.}
}

\answer[0 in]{
...
}

\vspace{12pt}

%----------------------------------------------------------------------

\noindent Use the master theorem (or main recurrence theorem if applicable) to solve the following recurrence relations. State which case of the theorem you are using and why.

\question[2]{
$T(n)=2T(\frac{n}{4})+1$
}

\answer[0 in]{
...
}

\question[2]{
$T(n)=2T(\frac{n}{4})+\sqrt{n}$
}

\answer[0 in]{
...
}

\question[2]{
$T(n)=2T(\frac{n}{4})+n$
}

\answer[0 in]{
...
}

\question[2]{
$T(n)=2T(\frac{n}{4})+n^2$
}

\answer[0 in]{
...
}
\end{document}
